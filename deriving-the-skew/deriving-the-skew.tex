\documentclass{article}
\usepackage {joyce-math}

\begin{document}
\title {Deriving $\lambda$ with the Method of Moments}
\author {Joyce Tipping}
\date{}
\maketitle

\section{Method of Moments}

Let $B \sim Bin(n, p)$ and $Y \sim SN(\xi, \omega^2, \lambda)$. We will find approximations for
$\xi$, $\omega$, and $\lambda$ by comparing the first, second, and third moments about the mean of
$B$ and $Y$.

\subsection{The Moments of the Binomial}

For $B$, our binomial, the first two moments, the mean and variance, are straightforward

\begin{align}
  E(B) &= np \\
  Var(B) &= np(1-p) \nonumber
\end{align}

From here, we can easily find

\begin{equation}
  E(B^2) = Var(B) + [E(B)]^2 = np(1-p) + n^2p^2 = np - np^2 + n^2p^2
\end{equation}

which we will need for the third moment. We will also need $E(B^3)$, for which we take a quick
detour through the third factorial moment:

\begin{align}
  E[B(B-1)(B-2)] &= \sum_{x=0}^n x (x-1) (x-2) \cdot \left\{ \binom{n}{x} p^x q^{n-x} \right\}
  \nonumber \\
  &= \sum_{x=3}^n x(x-1)(x-2) \cdot \frac{n!}{x!\;(n-x)!} \; p^x q^{n-x} \nonumber \\
  &= \sum_{x=3}^n \frac{n!}{(x-3)!\;(n-x)!} \; p^x q^{n-x} \nonumber \\
  &= \sum_{x=3}^n n(n-1)(n-2) p^3 \cdot \frac{(n-3)!}{(x-3)!\;(n-x)!} \; p^{x-3}q^{n-x} \nonumber \\
  \intertext{Let $y=x-3$. Then $x=y+3$, and $x=3 \rightarrow y=0$ and $x=n \rightarrow y=n-3$.}
  &= n(n-1)(n-2)p^3 \cdot \sum_{y=0}^{n-3} \frac{(n-3)!}{y!\;(n-(y+3))!} \; p^y q^{n-(y+3)} \nonumber \\
  &= n(n-1)(n-2)p^3 \cdot \underbrace {\sum_{y=0}^{n-3} \frac{(n-3)!}{y!\;((n-3)-y)!} \; p^y
    q^{(n-3)-y}}_{\mathclap{\text{[pdf of $Bin(n-3,p)$ summed from 0 to $n-3$] = 1}}} \nonumber \\
  &= n(n-1)(n-2)p^3 \nonumber \\
  &= n^3p^3 - 3n^2p^3 + 2np^3 \nonumber
  \intertext{Further expanding the left side and solving for $E(B^3)$,}
  E[B^3 - 3B^2 + 2B] &= n^3p^3 - 3n^2p^3 + 2np^3 \nonumber \\
  E(B^3) - 3E(B^2) + 2E(B) &= \nonumber \\
  E(B^3) - 3(np - np^2 + n^2p^2) + 2np &= \nonumber \\
  E(B^3) &= n^3p^3 - 3n^2p^3 + 2np^3 + 3np - 3np^2 + 3n^2p^2 - 2np \nonumber \\
  &= n^3p^3 - 3n^2p^3 + 2np^3 - 3np^2 + 3n^2p^2 + np
\end{align}

With these results (and a bit of elbow grease), we can easily obtain the third moment:

\begin{align}
  E([B - E(B)]^3) &= E(B^3 - 3B^2 E(B) + 3B [E(B)]^2 - [E(B)]^3) \nonumber \\
  &= E(B^3) - 3 E(B^2) E(B) + 3 E(B) [E(B)]^2 - [E(B)]^3 \nonumber \\
  &= E(B^3) - 3 E(B^2) E(B) + 2 [E(B)]^3 \nonumber \\
  &= (n^3p^3 - 3n^2p^3 + 2np^3 - 3np^2 + 3n^2p^2 + np) - 3np(np - np^2 + n^2p^2) + 2n^3p^3 \nonumber \\
  &= n^3p^3 - 3n^2p^3 + 2np^3 - 3np^2 + 3n^2p^2 + np - 3n^2p^2 + 3n^2p^3 - 3n^3p^3 + 2n^3p^3
  \nonumber \\
  &= 2np^3 - 3np^2 + np \nonumber \\
  &= np(p-1)(2p-1)
\end{align}

For our future convenience, we'll restate our three moments here:

\begin{align}
  E(B) &= np \nonumber \\
  E([B - E(B)]^2) &= np(1-p) \\
  E([B - E(B)]^3) &= np(p-1)(2p-1) \nonumber
\end{align}

\subsection{The Moments of the Skew Normal}

Now we'll take a look at the moments of the skew normal. According to Equation 1 in Pewsey (2000)

\begin{align}
  E(Y) &= \xi + \omega b \delta \nonumber \\
  E(Y^2) &= \xi^2 + 2\xi \omega b \delta + \omega^2 \\
  Var(Y) &= \omega^2 (1 - b^2 \delta^2) \nonumber \\
  E(Y^3) &= \xi^3 + 3 b \xi^2 \omega \delta + 3 \xi \omega^2 + 3 b \omega^3 \delta - b \omega^3
  \delta^3 \nonumber
\end{align}

where $b = \sqrt{\frac{2}{\pi}}$ and $\delta = \frac{\lambda}{\sqrt{1 + \lambda^2}}$.

Again, our first two moments are already taken care of. The third is a little more complicated:

\begin{align}
  E([Y - E(Y)]^3) &= E(Y^3) - 3E(Y^2)E(Y) + 2[E(Y)]^3 \nonumber \\
  &= (\xi^3 + 3 b \xi^2 \omega \delta + 3 \xi \omega^2 + 3 b \omega^3 \delta - b \omega^3 \delta^3)
  - 3 (\xi^2 + 2\xi \omega b \delta + \omega^2) (\xi + \omega b \delta) \nonumber \\
  & \quad + 2(\xi + \omega b \delta)^3 \nonumber \\
  &= \xi^3 + 3 b \xi^2 \omega \delta + 3 \xi \omega^2 + 3 b \omega^3 \delta - b \omega^3 \delta^3 -
  3 \xi^3 - 9 b \xi^2 \omega \delta - 6 b^2 \xi \omega^2 \delta^2 - 3 \xi \omega^2 \nonumber \\
  & \quad - 3 b \omega^3 \delta + 2 \xi^3 + 6 b \xi^2 \omega \delta + 6 b^2 \xi \omega^2 \delta^2 +
  2 b^3 \omega^3 \delta^3 \nonumber \\
  &= 2 \omega^3 b^3 \delta^3 - b \omega^3 \delta^3 \nonumber \\
  &= b \omega^3 \delta^3 (2b^2 - 1)
\end{align}

We restate our results:

\begin{align}
  E(B) &= \xi + b \omega \delta \nonumber \\
  E([B - E(B)]^2) &= \omega^2 (1 - b^2 \delta^2) \\
  E([B - E(B)]^3) &= b \omega^3 \delta^3 (2b^2 - 1) \nonumber
\end{align}

\subsection{Curiosity}

As a curiosity, I was unable to get Pewsey and Azzalini to agree with each other on
$E(Z^3)$. According to Pewsey (2000),

\begin{equation} \label{ey3-pewsey}
  E(Y^3) = \xi^3 + 3 b \xi^2 \omega \delta + 3 \xi \omega^2 + 3 b \omega^3 \delta - b \omega^3 \delta^3
\end{equation}

where $b = \sqrt{\frac{2}{\pi}}$ and $\delta = \frac{\lambda}{\sqrt{1 + \lambda^2}} \in (-1,
1)$. Since $Y = \xi + \omega Z$, by the linearity of expected value, we also have

\begin{align}
  E(Y^3) &= E [(\xi + \omega Z)^3] \nonumber \\
  &= E[\xi^3 + 3 \xi^2 \omega Z + 3 \xi \omega^2 Z^2 + \omega^3 Z^3] \nonumber \\
  &= \xi^3 + 3 \xi^2 \omega\;E(Z) + 3 \xi \omega^2\;E(Z^2) + \omega^3\;E(Z^3) \nonumber \\
  &= \xi^3 + 3 \xi^2 \omega\ b \delta + 3 \xi \omega^2 + \omega^3\;E(Z^3) \label{ey3-linear-expansion}
\end{align}

By comparing equations \ref{ey3-pewsey} and \ref{ey3-linear-expansion} and eliminating terms, we
arrive at

\begin{align}
  \omega^3\;E(Z^3) &= 3 b \omega^3 \delta - b \omega^3 \delta^3 \nonumber \\
  \Rightarrow \quad E(Z^3) &= 3 b \delta - b \delta^3 \nonumber \\
  &= b \delta (3 - \delta^2) \nonumber \\
  &= \sqrt{\frac{2}{\pi}} \cdot \frac{\lambda}{\sqrt{1 + \lambda^2}} \cdot \left( 3 -
    \frac{\lambda^2}{1 + \lambda^2} \right)
\end{align}

However, according to equation (6.5?) in Azzalini (2005),

\[
E(Z^r) =
 \begin{dcases*}
   1 \times 3 \times \cdots \times (r -1) & if r is even \\
   \frac{\sqrt{2}\;(2k + 1)!\;\lambda}{\sqrt{\pi}\;(1 + \lambda^2)^{k + 1/2}\;2^k}
   \sum_{m=0}^k\;\frac{m!\;(2\lambda)^{2m}}{(2m+1)!\;(k-m)!} & if $r = 2k+1$ and $k = 0, 1, ...$
\end{dcases*}
\]

So, for $E(Z^3)$, we have $r = 2k + 1 = 3$ and $k = 1$:

\begin{align}
  E(Z^3) &= \frac{\sqrt{2} \cdot 3! \cdot \alpha}{\sqrt{\pi} \cdot (1 + \alpha^2)^{3/2} \cdot 2} \;
  \sum_{m=0}^1 \frac{m!\;(2\alpha)^{2m}}{(2m + 1)!\;(1 - m)!} \nonumber \\
  &= \frac{3\sqrt{2}}{\sqrt{\pi}} \cdot \frac{\alpha}{(1 + \alpha^2)^{3/2}} \cdot \left( \frac{0!
      (2\alpha)^0}{1!1!} + \frac{1! (2\alpha)^2}{3!0!} \right) \nonumber \\
  &= \frac{3\sqrt{2}}{\sqrt{\pi}} \cdot \frac{\alpha}{(1 + \alpha^2)^{3/2}} \cdot \left( 1 +
    \frac{2}{3}\alpha^2 \right) \nonumber \\
  &= \sqrt{\frac{2}{\pi}} \cdot \frac{\alpha}{(\sqrt{1 + \alpha^2})^3} \cdot (3 + 2 \alpha^2)
\end{align}


\end{document}