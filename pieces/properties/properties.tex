\section{The Skew-Normal}

The skew-normal distribution is similar to the normal but with an added
parameter for skew. In this section, we'll introduce

\subsection{Basics}

Let $Y$ be a skew-normal distribution, with location parameter $\mu \in \R$,
scale parameter $\sigma > 0$, and shape parameter $\lambda \in \R$, denoted
$SN(\mu, \sigma, \lambda)$. Then $Y$ has pdf

\begin{equation} \label{eq:sn-pdf}
  f_Y(x) = \frac2\sigma \cdot \phi \left( \frac{x-\mu}{\sigma} \right) \cdot \Phi \left( \frac{\lambda(x-\mu)}{\sigma} \right)
\end{equation}

where $\phi$ is the standard normal pdf and $\Phi$ is the standard normal cdf.
Some other basic properties of $Y$, given by \citet{pewsey}, are

\begin{align}
  E(Y) &= \mu + b \delta \sigma \nonumber \\
  E(Y^2) &= \mu^2 + 2b \delta \mu \sigma + \sigma^2 \label{eq:sn-basic-properties} \\
  Var(Y) &= \sigma^2 (1 - b^2 \delta^2) \nonumber \\
  E(Y^3) &= \mu^3 + 3 b \delta \mu^2 \sigma + 3 \mu \sigma^2 + 3 b \delta \sigma^3 - b \delta^3 \sigma^3 \nonumber
\end{align}

where $b = \sqrt{\frac{2}{\pi}}$ and $\delta = \frac{\lambda}{\sqrt{1 +
\lambda^2}}$.

The $SN(0,1,\lambda)$ distribution is called the standard skew-normal, and has
pdf

\begin{equation} \label{eq:standard-sn-pdf}
  f_z(x) = 2 \cdot \phi(x) \cdot \Phi (-\lambda x)
\end{equation}

Like the normal and standard normal, you can arrive at the standard skew normal
by applying the transformation $\frac{Y - \mu}{\sigma}$.

A natural question to ask is how the skew-normal relates to the normal.
Fortunately, the connection is very intuitive: When $\lambda = 0$, Equation
\eqref{eq:sn-pdf} reverts to the normal pdf:

\begin{align*}
  f_Y(x|\lambda=0) &= \frac2\sigma \cdot \phi \left( \frac{x-\mu}{\sigma} \right) \cdot \Phi(0) \\
  &= \frac2\sigma \cdot \phi \left( \frac{x-\mu}{\sigma} \right) \cdot 0.5 \\
  &= \frac1\sigma \cdot \phi \left( \frac{x-\mu}{\sigma} \right) \\
  &= \frac1\sigma \cdot \frac{1}{\sqrt{2\pi}} \;\cdot\; \exp \left( -\frac{(x-\mu)^2}{2\sigma^2} \right) \\
  &= \frac{1}{\sqrt{2\pi\sigma^2}} \;\cdot\; \exp \left( -\frac{(x-\mu)^2}{2\sigma^2} \right)
\end{align*}

\subsection{Four Properties}
\label{subsec:four-properties}

\citet{main} gives four handy properties of the skew-normal distribution:

\begin{property} \label{prop:1}
  If $Z \sim SN(0, 1, \lambda)$, then $(-Z) \sim SN(0, 1, -\lambda)$.
\end{property}

\begin{proof}
  The standard normal pdf is an even function: $\phi(-x) =
  \frac{1}{\sqrt{2\pi}}\;e^{-(-x)^2/2} = \frac{1}{\sqrt{2\pi}}\;e^{-x^2/2} =
  \phi(x)$. But the standard normal cdf, \thinspace $\Phi(x) =
  \int_{-\infty}^\infty \phi(x)$, \thinspace is not, being 0 near $-\infty$ and
  1 near $\infty$. Thus,
  
  \begin{align*}
    f_{(-Z)}(x) &= f_Z(-x) \\
    & = 2 \cdot \phi(-x) \cdot \Phi (-\lambda x) \\
    & = 2 \cdot \phi(x) \cdot \Phi (-\lambda x)
  \end{align*}

  which is the pdf of $SN(0, 1, -\lambda)$.
\end{proof}

\begin{property} \label{prop:2}
  As $\lambda \to \pm \infty$, \thinspace $SN(0,1,\lambda)$ tends to the half normal distribution, $\pm |N(0,1)|$.
\end{property}

To prove our theorem, it is helpful to formally define the half normal distribution:

\begin{helper-lem} \label{lem:p2-half-normal}
  Let $X \sim N(0, \sigma^2)$. Then the distribution of $|X|$ is a half-normal
  random variable with parameter $\sigma$ and

  \begin{equation*}
    f_{|X|}(x) =
    \begin{dcases*}
      2 \cdot f_X(x) & when $0 < x < \infty$ \\
      0 & everywhere else
    \end{dcases*}
  \end{equation*}
\end{helper-lem}

\begin{proof}
  Let $X \sim N(0, \sigma^2)$, defined over $A = (-\infty, \infty)$. Define 
  
  \begin{equation*}
    Y = |X| =
    \begin{dcases*}
      -x & if $x < 0$ \\
      0 & if $x = 0$ \\
      x & if $x > 0$ \\
    \end{dcases*}
  \end{equation*}

  $Y$ is not one-to-one over $A$. However, we can partition $A$ into disjoint
  subsets $A_1 = (-\infty, 0)$, $A_2 = (0, \infty)$, and $A_3 = \{0\}$ such
  that $A = A_1 \cup A_2 \cup A_3$ and $Y$ is one-to-one over each $A_i$. We
  can then transform each piece separately using Theorem 6.3.2 from
  \citet{textbook}:

  On $A_1$: $y = -x \lra x = -y$ and $\J = \left| \frac{dx}{dy} \right| = |-1|
  = 1$, yielding

  \begin{align}
    f_Y(y) &= f_X(x) \cdot \J \nonumber \\
    &= f_X(-y) \cdot 1 \nonumber \\
    &= \frac{1}{\sqrt{2\pi}\sigma} \; e^{-\frac{(-y)^2}{2\sigma}} \nonumber \\
    &= \frac{1}{\sqrt{2\pi}\sigma} \; e^{-\frac{y^2}{2\sigma}} \nonumber \\
    &= f_X(y) \label{eq:p2-transform-a1}
  \end{align}

  over the domain $A_1: -\infty < x < 0 \lra -\infty < -y < 0 \lra 0 < y < \infty :B_1$.

  Similarly, on $A_2$: $y = x \lra x = y$ and $\J = \left| \frac{dx}{dy}
  \right| = |1| = 1$, yielding

  \begin{align}
    f_Y(y) &= f_X(x) \cdot \J \nonumber \\
    &= f_X(y) \cdot 1 \nonumber \\
    &= f_X(y) \label{eq:p2-transform-a2}
  \end{align}

  over the domain $A_2: 0 < x < \infty \lra 0 < y < \infty :B_2$.

  On $A_3$, we have $x = 0, y = 0$ and $\J = \left| \frac{dx}{dy} \right| = |0|
  = 0$, yielding $f_Y(y) = f_X(x) \cdot \J = f_X(x) \cdot 0 = 0$.

  Now, by Theorem 6.3.10 from \citet{textbook}, we achieve our result by simply
  summing \eqref{eq:p2-transform-a1} and \eqref{eq:p2-transform-a2}.

  \begin{align}
    f_Y(y) &= \{ f_Y(y) \textrm{ over } A_1 \} + \{ f_Y(y) \textrm{ over } A_2 \} \nonumber \\
    &= f_X(y) + f_X(y) \nonumber \\
    &= 2 \cdot f_X(y) \label{eq:p2-transform-result}
  \end{align}

  over $B = B_1 \cup B_2 = (0, \infty)$, and 0 otherwise.
\end{proof}

With Lemma \ref{lem:p2-half-normal}, we can easily show our property:

\begin{proof}[Proof of Property 2]
  Let $Z \sim SN(0,1,\lambda)$. Recall that $f_z(x) = 2 \cdot \phi(x) \cdot
  \Phi(\lambda x)$.

  Consider $\lim_{\lambda \to \infty} f_X(x)$. When $x$ is negative, $\lambda x
  \to -\infty$ and thus $\Phi(\lambda x) \to 0$. When $x$ is positive, however,
  $\lambda x \to \infty$ and $\Phi(\lambda x) \to 1$. Thus

  \begin{equation}
    \label{eq:p2-positive-half-normal}
    \lim_{\lambda \to \infty} 2 \cdot \phi(x) \cdot \Phi(\lambda x) \eq
    \begin{dcases*}
      0 & when $x \leq 0$ \\
      2 \cdot \phi(x) & when $x > 0$
    \end{dcases*}
    \eq |N(0,1)|
  \end{equation}

  In $\lim_{\lambda \to -\infty} f_X(x)$, the signs are reversed. When $x$ is
  negative, $\lambda x \to \infty$ and $\Phi(\lambda x) \to 1$. When $x$ is
  positive, $\lambda x \to -\infty$ and $\Phi(\lambda x) \to 0$. Thus,

  \begin{equation}
    \label{eq:p2-negative-half-normal}
    \lim_{\lambda \to -\infty} 2 \cdot \phi(x) \cdot \Phi(\lambda x) \eq
    \begin{dcases*}
      2 \cdot \phi(x) & when $x < 0$ \\
      0 & when $x \geq 0$
    \end{dcases*}
    \eq -|N(0,1)|
  \end{equation}
\end{proof}

\begin{property} \label{prop:3}
  If $Z \sim SN(0, 1, \lambda)$, then $Z^2 \sim \chi^2_1$ (chi-square with 1 degree of freedom).
\end{property}

\begin{proof}
  To prove our result, we make use of Lemma 1 in \citet{azzalini}:

  \begin{helper-lem}
    If $f_0$ is a one-dimensional probability density function symmetric about
    0, and $G$ is a one-dimensional distribution function such that $G'$ exists
    and is a density symmetric about 0, then

    \begin{equation}
      \label{eq:azzalini-perturbation-invariance}
      f(z) = 2 \cdot f_0(z) \cdot G\{w(z)\} \quad (-\infty < z < \infty)
    \end{equation}

    is a density function for any odd function $w(\cdot)$.
  \end{helper-lem}

  Notice that $\phi(x)$ is a one-dimensional probability density function
  symmetric about 0, and $\Phi(x)$ is a one-dimensional distribution function
  such that $\Phi'$ exists and is a density symmetric about 0. Furthermore,
  $\lambda x$ is an odd function. Thus, $f_z(z) = 2 \cdot \phi(z) \cdot
  \Phi(\lambda z)$ conforms to equation
  \eqref{eq:azzalini-perturbation-invariance}. With that in mind, the corollary
  to this lemma provides a very useful result:

  \begin{helper-cor}[Perturbation Invariance]
    If $Y \sim f_0$ and $Z \sim f$, then $|Y| \overset{d}{=} |Z|$, where the
    notation $\overset{d}{=}$ denotes equality in distribution.    
  \end{helper-cor}

  Thus, we can treat $\phi$ and $Z$ as being equal in distribution. We will now
  show that $\phi^2 \sim \chi^2_1$:

  \begin{align*}
    M_{\phi^2}(t) &= E[e^{tx^2}] \\
    &= \int_{-\infty}^\infty \; e^{tx^2} \left[ \frac{1}{\sqrt{2\pi}} e^{-x^2/2} \right] dx \\
    &= \int_{-\infty}^\infty \; \frac{1}{\sqrt{2\pi}} \; e^{tx^2 - x^2/2} \; dx \\
    &= \int_{-\infty}^\infty \; \frac{1}{\sqrt{2\pi}} \; e^{-\frac{x^2}{2}(1 - 2t)} \; dx \\
    &= \int_{-\infty}^\infty \; \frac{1}{\sqrt{2\pi}} \; e^{-\frac{1}{2}(\sqrt{1-2t} \; x)^2} \; dx \\
    \intertext{Let $u = (\sqrt{1-2t}) \, x$; then we have $du = \sqrt{1-2t}$, \enspace $dx = \frac{du}{\sqrt{1-2t}}$, \enspace and our limits become $x \to -\infty \Ra u \to -\infty$
      and $x \to \infty \Ra u \to \infty$.}
    &= \int_{-\infty}^\infty \; \frac{1}{\sqrt{2\pi}} \; e^{-u^2/2} \; \left( \frac{1}{\sqrt{1-2t}} \right) du \\
    &= \frac{1}{\sqrt{1-2t}} \; \underbrace{\left( \int_{-\infty}^\infty \; \frac{1}{\sqrt{2\pi}} \; e^{-u^2/2} \; du \right)}_{\mathclap{\textnormal{$\phi(u)$ integrated over
      $(-\infty,\infty)$ = 1}}} \label{phi-pdf} \\
    &= \frac{1}{\sqrt{1-2t}}
  \end{align*}

  which is the MGF of the $\chi^2_1$. Since $Z$ is equal in distribution to
  $\phi$, we can also conclude that $Z^2 \sim \chi^2_1$. \end{proof}

\begin{property} \label{prop:4}
  The MGF of $SN(0,1,\lambda)$ is

  \begin{equation} \label{eq:p4-sn-mgf}
    M(t|\lambda) = 2 \cdot \Phi (\delta t) \cdot e^{t^2/2}
  \end{equation}
    
  where $\delta = \frac{\lambda}{\sqrt{1 + \lambda^2}}$ and $t \in (-\infty, \infty)$.
\end{property}

\begin{proof}
  According to Equation 5 in \citet{azzalini}, the MGF of $SN(\mu, \sigma^2,
  \lambda)$ is

  \begin{equation*}
    M(t) \eq E\{e^{tY}\} \eq 2 \cdot \exp \left( \mu t + \frac{\sigma^2 t^2}{2} \right) \cdot \Phi(\delta \sigma t)
  \end{equation*}

  where $\delta = \frac{\lambda}{1 + \lambda^2} \in (-1, 1)$. It follows that
  the MGF of the $SN(0, 1, \lambda)$ is

  \begin{equation*}
    2 \cdot \exp \left( 0 \cdot t + \frac{1 \cdot t^2}{2} \right) \cdot \Phi(\delta \cdot 1 \cdot t) \eq 2 \cdot e^{t^2/2} \cdot \Phi(\delta t)
  \end{equation*}
\end{proof}
