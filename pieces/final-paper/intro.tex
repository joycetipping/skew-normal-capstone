\title{The Skew-Normal Approximation of the Binomial Distribution}
\author{Joyce Tipping}
\date{2010}
\maketitle

\section{Introduction}

One of the most basic distributions in statistics is the binomial, $X \sim
Bin(n,p)$, with pdf

\begin{equation*}
  f_X(x) = \binom{n}{x} p^x q^{n-x}
\end{equation*}

Calculating the binomial cdf, $F_X(x) = P(X \leq x) = \sum_{k=1}^x f_X(k)$, by
hand is manageable for small $n$ but quickly becomes cumbersome as $n$ grows
even mediumly large. A common strategy is to use the normal distribution as an
approximation:

\begin{equation}
  F_X(x) \approx \Phi \left( \frac{k + 0.5 - \mu}{\sigma} \right)
\end{equation}

where $\Phi$ is the standard normal cdf and $\mu = np$ and $\sigma =
\sqrt{np(1-p)}$.

This approximation works well when either $n$ is very large (invoking the
Central Limit Theorem) or when $p$ is close to 0.5 (making $X$ roughly
symmetric). These rules of thumb are captured in a requirement, often stated as

\begin{equation*}
  np(1-p) > 9
\end{equation*}

or

\begin{align*}
      np &> 5 \quad \textnormal{for} \quad  0 < p \leq 0.5, \\
  n(1-p) &> 5 \quad \textnormal{for} \quad  0.5 < p < 1
\end{align*}

However, when $n$ is not large and $p$ is close to 0 or 1, the binomial
distribution is skewed and even if the above requrements are met, the
approximation can be inaccurate; sometimes, as demonstrated by \citet{mabs},
rather substantially. In these cases, the skew-normal approximation can provide
an alternative -- and considerably more accurate -- method of approximation.
