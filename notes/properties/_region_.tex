\message{ !name(properties.tex)}\documentclass{article}
\usepackage{joyce-math}

\begin{document}

\message{ !name(properties.tex) !offset(-3) }

\title{Update 1}
\date{}
\maketitle

The four properties of the skew normal listed in section 2 of ``A Note on Improved Approx \ldots''.

\begin{itemize}
\item Property 1
  \begin{thm*}
    If $Z \sim SN(0, 1, \lambda)$, then $(-Z) \sim SN(0, 1, -\lambda)$.
  \end{thm*}
  
  \begin{proof}

    The standard normal pdf is an even function: $\phi(-x) =
    \frac{1}{\sqrt{2\pi}}\;e^{-\frac{(-x)^2}{2}} = \frac{1}{\sqrt{2\pi}}\;e^{-x^2/2} =
    \phi(x)$. However, $\Phi(x) \eq \int_{-\infty}^\infty \phi(x)$ is odd. Thus,

    \begin{align*}
      f_{(-Z)}(x) &= f_Z(-x) \\
      & = 2 \cdot \phi(-x) \cdot \Phi (-\lambda x) \\
      & = 2 \cdot \phi(x) \cdot \Phi (-\lambda x)
    \end{align*}
    which is the pdf of $SN(0, 1, -\lambda)$.
  \end{proof}

\item Property 2
  \begin{thm*}
    As $\lambda \to \pm \infty$, $SN(0, 1, \lambda)$ tends to the half normal distribution: the $\pm
    |N(0,1)|$ distribution.
  \end{thm*}

  To prove our theorem, it is helpful to have on hand a formal pdf for the half normal distribution.

  \begin{lemma*}
    Let $X \sim N(0, \sigma^2)$. Then the distribution of $|X|$ is a half-normal random variable
    with parameter $\sigma$ and $f_{|X|}(x) = 2 \cdot \phi(x)$ when $0 < x < \infty$, and 0
    everywhere else.
  \end{lemma*}

  \begin{proof}
    Let $X \sim N(0, \sigma^2)$, defined over $A = (-\infty, \infty)$.

    Define 
    
    \[
    Y = |X| =
    \begin{dcases*}
      -x & if $x < 0$ \\
      0 & if $x = 0$ \\
      x & if $x > 0$ \\
    \end{dcases*}
    \]

    $Y$ is not one-to-one over $A$. However, we can partition $A$ into disjoint
    subsets $A_1 = (-\infty, 0)$, $A_2 = (0, \infty)$, and $A_3 = \{0\}$ such that $A = A_1 \cup A_2
    \cup A_3$ and $Y$ is one-to-one over each $A_i$. We can then transform each piece separately
    using Theorem 6.3.2:

    On $A_1$: $y = -x \lra x = -y$ and $\J = \left| \frac{dx}{dy} \right| = |-1| = 1$, yielding
    \begin{align*}
      f_Y(y) &= f_x(x) \cdot \J\\
      &= f_x(-y) \cdot 1 \\
      &= \frac{1}{\sqrt{2\pi}\sigma} \; e^{-\frac{(-y)^2}{2\sigma}} \\
      &= \frac{1}{\sqrt{2\pi}\sigma} \; e^{-\frac{y^2}{2\sigma}} \\
      &= \phi(y)
    \end{align*}
    over the domain $A_1: -\infty < x < 0 \lra -\infty < -y < 0 \lra 0 < y < \infty :B_1$.

    Similarly, on $A_2$: $y = x \lra x = y$ and $\J = \left| \frac{dx}{dy} \right| = |1| = 1$,
    yielding
    \begin{align*}
      f_Y(y) &= f_x(x) \cdot \J\\
      &= f_x(y) \cdot 1 \\
      &= \frac{1}{\sqrt{2\pi}\sigma} \; e^{-\frac{y^2}{2\sigma}} \\
      &= \phi(y)
    \end{align*}
    over the domain $A_2: 0 < x < \infty \lra 0 < y < \infty :B_2$.

    Since $y$ is 0 on $A_3$, it can be excluded from our transformation.

    Then, by Theorem 6.3.10,
    \begin{align*}
      f_Y(y) &= \{ f_Y(y) \textrm{ over } A_1 \} + \{ f_Y(y) \textrm{ over } A_2 \} \\
      &= \phi(y) + \phi(y) \\
      &= 2 \cdot \phi(y)
    \end{align*}
    over $B = B_1 \cup B_2 = (0, \infty)$.    
  \end{proof}

  With this result, we can easily show our property:

  \begin{proof}
    Let $X \sim SN(0,1,\lambda)$. Then $f_X(x) = 2 \cdot \phi(x) \cdot \Phi(\lambda x)$.

    Consider $\lim_{\lambda \to \infty} f_X(x)$. When $x$ is negative, $\lambda x \to -\infty$ and
    thus $\Phi(\lambda x) \to 0$. When $x$ is positive, however, $\lambda x \to \infty$ and
    $\Phi(\lambda x) \to 1$. Thus we have

    \[
    \lim_{\lambda \to \infty} 2 \cdot \phi(x) \cdot \Phi(\lambda x) =
    \begin{dcases*}
      0 & when $x < 0$ \\
      2 \cdot \phi(x) & when $x \geq 0$
    \end{dcases*}
    \]

    In $\lim_{\lambda \to -\infty} f_X(x)$, the signs are reversed. When $x$ is negative, $\lambda x
    \to \infty$ and $\Phi(\lambda x) \to 1$. When $x$ is positive, $\lambda x \to -\infty$ and
    $\Phi(\lambda x) \to 0$. Thus,

    \[
    \lim_{\lambda \to -\infty} 2 \cdot \phi(x) \cdot \Phi(\lambda x) =
    \begin{dcases*}
      2 \cdot \phi(x) & when $x \leq 0$ \\
      0 & when $x > 0$
    \end{dcases*}
    \]

  \end{proof}

\item Property 3
  \begin{thm*}
    If $Z \sim SN(0, 1, \lambda)$, then $Z^2 \sim \chi^2_1$ (chi-square with 1 df).
  \end{thm*}

  \begin{proof}
    To prove our result, we make use of a lemma in Azzalini (2005):

    \begin{lemma}
      If $f_0$ is a one-dimensional probability density function symmetric about 0, and $G$ is a
      one-dimensional distribution function such that $G'$ exists and is a density symmetric about
      0, then
      \[
      f(z) = 2 f_0(z) G\{w(z)\} \quad (-\infty < z < \infty)
      \]
    \end{lemma}
    Notice that $phi(x)$ is a one-dimensional probability density function
    \[
    M_{Z^2}(t) = E[e^{tZ^2}] = 2 \; \int_{-\infty}^\infty \; e^{tZ^2} \cdot \phi (z) \cdot \Phi
    (\lambda z) \; dz \quad = \quad ?
    \]
  \end{proof}

\item Property 4
  \begin{thm*}
    The MGF of $SN(0, 1, \lambda)$ is $M(t|\lambda) = 2 \cdot \Phi (\delta t) \cdot e^{t^2/2}$ where
    $\delta = \frac{\lambda}{\sqrt{1 + \lambda^2}}$ and $t \in (-\infty, \infty)$.
  \end{thm*}

  \[
  2 \int_{-\infty}^\infty \; e^{tx} \cdot \phi (x) \cdot \Phi (\lambda x) \; dx \quad = \quad ?
  \]

\end{itemize}
\end{document}
\message{ !name(properties.tex) !offset(-164) }
