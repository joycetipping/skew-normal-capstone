\section{Solving for $\lambda$}
\label{sec:solving-for-lambda}

Hoping that it would provide some insight into \eqref{eq:lambda-solved}, I
expanded \eqref{eq:solving-for-lambda} after moving all terms to the left hand
side:

\begin{gather*}
  \left. \left( 1 - \frac{2}{\pi} \cdot \frac{\lambda^2}{1+\lambda^2} \right)^3 \middle/ \left[ \frac{2}{\pi} \left( \frac{\lambda^2}{1+\lambda^2} \right)^3 \left( \frac{4}{\pi} - 1
    \right)^2 \right] \right. \;-\; \frac{np(1-p)}{(1-2p)^2} \\
  \left. \left( \frac{\pi(1+\lambda^2) - 2\lambda^2}{\pi(1+\lambda^2)} \right)^3 \middle/ \left( \frac{2(4-\pi)^2\lambda^6} {\pi^3(1+\lambda^2)^3} \right) \right. \;-\; \frac{np(1-p)}{(1-2p)^2} \\
  \left( \frac{\pi^3(1+\lambda^2)^3 - 3\pi^2(1+\lambda^2)^2 \cdot 2\lambda^2 + 3\pi(1+\lambda^2) \cdot 4\lambda^4 - 8\lambda^6}{\cancel{\pi^3 (1+\lambda^2)^3}} \right) \cdot \left(
    \frac{\cancel{\pi^3(1+\lambda^2)^3}}{2(4-\pi)^2\lambda^6} \right) \;-\; \frac{np(1-p)}{(1-2p)^2} \\
  \frac{\pi^3(1+3\lambda^2+3\lambda^4+\lambda^6) - 6\pi^2\lambda^2(1+2\lambda^2+4\lambda^4) + 12\pi\lambda^4(1+\lambda^2) - 8\lambda^6}{2(4-\pi)^2\lambda^6} \;-\; \frac{np(1-p)}{(1-2p)^2} \\
  \frac{\pi^3 + 3\pi^3\lambda^2 + 3\pi^3\lambda^4 + \pi^3\lambda^6 - 6\pi^2\lambda^2 - 12\pi^2\lambda^4 - 24\pi^2\lambda^6 + 12\pi\lambda^4 + 12\pi\lambda^6 - 8\lambda^6}{2(4-\pi)^2\lambda^6}
    \;-\; \frac{np(1-p)}{(1-2p)^2} \\
  \frac{\lambda^6(\pi^3-24\pi^2+12\pi-8) + \lambda^4(3\pi^3-12\pi^2+12\pi) + \lambda^2(3\pi^3-6\pi^2) + \pi^3}{2(4-\pi)^2\lambda^6} \;-\; \frac{np(1-p)}{(1-2p)^2}
  \intertext{Taking $c_1 = \pi^3 - 24\pi^2 + 12\pi - 8$, $c_2 = 3\pi^3 - 12\pi^2 + 12\pi$, $c_3 = 3\pi^3 - 6\pi^2$, $c_4 = \pi^3$, and $c_5 = 2(4-\pi)^2$, we can simplify this to}
  \frac{c_1\lambda^6 + c_2\lambda^4 + c_3\lambda^2 + c_4}{c_5\lambda^6} - \frac{np(1-p)}{(1-2p)^2} \\
  \frac{c_1(1-2p)^2\lambda^6 + c_2(1-2p)^2\lambda^4 + c_3(1-2p)^2\lambda^2 + c_4(1-2p)^2 - c_5\;np(1-p)\lambda^6}{c_5(1-2p)^2\lambda^6} \\
  \frac{\left[c_1(1-2p)^2 - c_5\;np(1-p)\right]\lambda^6 + \left[c_2(1-2p)^2\right]\lambda^4 + \left[c_3(1-2p)^2\right]\lambda^2 + c_4(1-2p)^2}{c_5(1-2p)^2\;\lambda^6}
\end{gather*}

\section{Curiosity}

As a curiosity, I was unable to get Pewsey and Azzalini to agree with each
other on $E(Z^3)$. According to Pewsey (2000),

\begin{equation} \label{eq:ey3-pewsey}
  E(Y^3) = \mu^3 + 3 b \delta \mu^2 \sigma + 3 \mu \sigma^2 + 3 b \delta \sigma^3 - b \delta^3 \sigma^3
\end{equation}

where $b = \sqrt{\frac{2}{\pi}}$ and $\delta = \frac{\lambda}{\sqrt{1 +
\lambda^2}} \in (-1, 1)$. Since $Y = \mu + \sigma Z$, by the linearity of
expected value, we also have

\begin{align}
  E(Y^3) &= E \left[ (\mu + \sigma Z)^3 \right] \nonumber \\
  &= E (\mu^3 + 3 \mu^2 \sigma Z + 3 \mu \sigma^2 Z^2 + \sigma^3 Z^3) \nonumber \\
  &= \mu^3 + 3 \mu^2 \sigma\;E(Z) + 3 \mu \sigma^2\;E(Z^2) + \sigma^3\;E(Z^3) \nonumber \\
  &= \mu^3 + 3 b \delta \mu^2 \sigma + 3 \mu \sigma^2 + \sigma^3\;E(Z^3) \label{eq:ey3-linear-expansion}
\end{align}

By comparing equations \eqref{eq:ey3-pewsey} and
\eqref{eq:ey3-linear-expansion} and eliminating terms, we arrive at

\begin{align}
  \sigma^3\;E(Z^3) &= 3 b \delta \sigma^3 - b \delta^3 \sigma^3 \nonumber \\
  \Rightarrow \quad E(Z^3) &= 3 b \delta - b \delta^3 \nonumber \\
  &= b \delta (3 - \delta^2) \nonumber \\
  &= \sqrt{\frac{2}{\pi}} \cdot \frac{\lambda}{\sqrt{1 + \lambda^2}} \cdot \left( 3 - \frac{\lambda^2}{1 + \lambda^2} \right) \label{eq:ez3-pewsey-derived}
\end{align}

However, according to equation (6.5?) in Azzalini (2005),

\begin{equation}
  E(Z^r) =
  \begin{dcases*}
    1 \times 3 \times \cdots \times (r -1) & if r is even \\
    \frac{\sqrt{2}\;(2k + 1)!\;\lambda}{\sqrt{\pi}\;(1 + \lambda^2)^{k + 1/2}\;2^k} \sum_{m=0}^k\;\frac{m!\;(2\lambda)^{2m}}{(2m+1)!\;(k-m)!} & if $r = 2k+1$ and $k = 0, 1, ...$
  \end{dcases*}
\end{equation}

So, for $E(Z^3)$, we have $r = 2k + 1 = 3$ and $k = 1$:

\begin{align}
  E(Z^3) &= \frac{\sqrt{2} \cdot 3! \cdot \lambda}{\sqrt{\pi} \cdot (1 + \lambda^2)^{3/2} \cdot 2} \; \sum_{m=0}^1 \frac{m!\;(2\lambda)^{2m}}{(2m + 1)!\;(1 - m)!} \nonumber \\
  &= \frac{3\sqrt{2}}{\sqrt{\pi}} \cdot \frac{\lambda}{(1 + \lambda^2)^{3/2}} \cdot \left( \frac{0! (2\lambda)^0}{1!1!} + \frac{1! (2\lambda)^2}{3!0!} \right) \nonumber \\
  &= \frac{3\sqrt{2}}{\sqrt{\pi}} \cdot \frac{\lambda}{(1 + \lambda^2)^{3/2}} \cdot \left( 1 + \frac{2}{3}\lambda^2 \right) \nonumber \\
  &= \sqrt{\frac{2}{\pi}} \cdot \frac{\lambda}{(\sqrt{1 + \lambda^2})^3} \cdot (3 + 2 \lambda^2) \label{eq:ez3-azzalini}
\end{align}

Unfortunately, equations \eqref{eq:ez3-pewsey-derived} and
\eqref{eq:ez3-azzalini} do not really line up.
