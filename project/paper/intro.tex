\title{The Skew-Normal Approximation of the Binomial Distribution}
\author{Joyce Tipping}
\date{2010}
\maketitle

\section{Introduction}

One of the most basic distributions in probability and statistics is the
binomial, $X \sim Bin(n,p)$, $n = 1, 2, \ldots$, $p \in (0, 1)$ with
probability density function (pdf)

\begin{equation*}
  f_X(x) = \binom{n}{x} p^x q^{n-x}, \quad x = 0, 1, \ldots, n,
\end{equation*}

where $q=1-p$. Calculating the binomial cumulative distribution function (cdf),
$F_X(x) = P(X \leq x) = \sum_{k=0}^x f_X(k)$, by hand is manageable for small
$n$ but quickly becomes cumbersome as $n$ grows even moderately large. A common
strategy is to use the normal distribution\footnotemark as an approximation:

\footnotetext{A random variable $X$ follows the normal distribution with mean
$\mu$ and variance $\sigma^2$ (denoted $X \sim N(\mu,\sigma^2)$) if it has the
pdf $f(x;\mu,\sigma) = \frac{1}{\sqrt{2\pi}\,\sigma} \;
e^{-[(x-\mu)/\sigma]^2/2}$ for $-\infty < x < \infty$, where $-\infty < \mu <
\infty$ and $0 < \sigma < \infty$.

$N(0,1)$ is an important special case known as the standard normal (denoted
$Z$) and has pdf $\phi(z) = \frac{1}{\sqrt{2\pi}} \; e^{-z^2/2}, \quad -\infty
< z < \infty$ and cdf given by $\Phi(z) = \int_{-\infty}^z \phi(t)\;dt$.
Equations 3.3.27, 3.3.29, and 3.3.30, \citet{textbook}.}

\begin{equation}
  F_X(x) \approx \Phi \left( \frac{x + 0.5 - \mu}{\sigma} \right),
\end{equation}

where $\Phi$ is the standard normal cdf and $\mu = np$ and $\sigma =
\sqrt{np(1-p)}$.

This approximation works best when the binomial is perfectly symmetrical, with
$p=0.5$. However as $p$ travels away from 0.5 in either direction, the binomial
becomes increasingly skewed, and one must either provide larger values of $n$
to compensate or face growing inaccuracy.

To protect against these cases, many textbooks impose a condition on the normal
approximation, often stated as either (1) $np(1-p) > 9$ \; or \; (2) $np > 5$
for $0 < p \leq 0.5$, and $n(1-p) > 5$ for $0.5 < p < 1$. However, \citet{mabs}
show that even when these conditions are met, the inaccuracy of the normal
approximation can be surprisingly large. So, the natural question arises: Can
we do better?

Perhaps so. Intuitively speaking, the main problem with the normal
approximation is that the normal is always symmetric while the binomial is
skewed for all values of $p$ not equal to 0.5. This fact suggests using the
skew-normal distribution, which adds an extra parameter to capture an
asymetrical lean.

In this paper, we will explore the aptness of the skew-normal as a method of
approximating the binomial. First, we'll acquaint ourselves with its basic
properties (section \ref{sec:properties}). We'll then derive an approximation
via the method of moments (section \ref{sec:method-of-moments}) and compare its
accuracy to that of the normal approximation (section \ref{sec:accuracy}).
Finally, we will leave the reader with a few practical resources for applying
our new method (section \ref{sec:resources}).
