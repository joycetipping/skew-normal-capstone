\section{Demonstrating Improved Accuracy}
\label{sec:accuracy}

Now comes the time to justify our efforts by comparing the accuracy of our
skew-normal approximation to that of the normal. Naturally, we are primarily
interested in cases where the normal approximation performs poorly, that is
binomials where $n$ is small and $p$ is either close to 0 or 1.\footnotemark

\footnotetext{Again, we examine only $p \in (0, 0.5)$. (See footnote
\ref{fnote:half-p-range}.)}

\subsection{Visual Comparison}

The first and most obvious way of judging accuracy is by visual inspection.
Figures \ref{fig:comparison-n25}, \ref{fig:comparison-n50}, and
\ref{fig:comparison-n100} compare the binomial, normal, and skew-normal at
small values of $p$ for $n=25$, $n=50$, and $n=100$, respectively.

At very small $n$ and $p$, our skew-normal curve follows the shape of the
binomial much more closely than the normal. However, as $n$ grows, the Central
Limit Theorem begins to exert its effect; even at a moderate $n = 100$ and a
smallish $p = 0.2$, the normal and skew-normal become nearly identical. Thus,
as we would expect, the skew-normal method is most valuable at small $n$ and
extreme $p$.

\subsection{Maximal Absolute Error}
\label{subsec:mabs}

Another more quantitative method of judging accuracy is comparing the maximal
absolute errors of our two approximations, defined by \citet{mabs} as

\begin{equation}
  \textnormal{MABS}(n, p) \eq \max_{k \in \{0, 1,...,n\}} \left| F_{B(n,p)} (k) -  F_{\textnormal{appr}(n,p)}(k + 0.5) \right|
\end{equation}

where $F_{B(n,p)}$ is the cdf of the binomial and $F_{\textnormal{appr}(n,p)}$
is the cdf of either the normal or skew-normal approximation; the 0.5 is a
continuity correction.

Figures \ref{fig:mabs-fixed-n} and \ref{fig:mabs-fixed-p} shows the MABS of the
skew-normal and normal approximations as a function of $p$ and $n$,
respectively. Again, the skew-normal outperforms the normal considerably in the
extreme ranges, with the two approximations converging as $n \rightarrow
\infty$ or $p \rightarrow 0.5$.
