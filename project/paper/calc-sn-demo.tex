\section{Calculating a Skew-Normal Approximation}
\label{app:calc-sn-approx}

Although easier with a computer program, calculating estimates for $\mu$,
$\sigma$, and $\lambda$ by hand is perfectly possible. Here, we will
demonstrate using $n=25$, $p=0.1$.

By far the largest battle is finding $\lambda$. We will use Equation
\eqref{eq:solving-for-lambda} but with the simplified left hand side given by
\eqref{eq:lhs-solving-for-lambda}:

\begin{equation}
  \label{eq:calculating-lambda}
  \left( \frac{1+\lambda^2}{\lambda^2} - \frac{2}{\pi} \right)^3 \left( \frac{\pi^3}{2(4-\pi)^2} \right) = \frac{np(1-p)}{(1-2p)^2} \;.
\end{equation}

The closed-formed solution to this equation is long, hideous, and hard to work
with, so for this demonstration, we will take a numerical approach.

The left hand side of \eqref{eq:calculating-lambda} is a function of lambda;
let us denote it $f(\lambda)$. The right hand side is a constant in $n$ and
$p$; let us call it $k_{n,p}$. Our goal is to find a value of $\lambda$ such
that $f(\lambda)$ is within a certain margin of error, $e$, of $k_{n,p}$. Since
we are computing by hand, we will take $e$ to be a modest 0.01.

Recall that the sign of $\lambda$ is determined independently of the value. In
fact, $f$ is never affected by the sign of $\lambda$, as all terms are squared.
Thus, we can restrict our search for $\lambda$ to the interval $(0, \infty)$.

Next, by taking $f$'s derivative, we can show that it is monotonically
decreasing for positive $\lambda$:

\begin{align*}
  \frac{d}{d\lambda} \left[ \left( \frac{1+\lambda^2}{\lambda^2} - \frac{2}{\pi} \right)^3 \left( \frac{\pi^3}{2(4-\pi)^2} \right) \right] &= \left( \frac{\pi^3}{2(4-\pi)^2} \right) \cdot
    3 \left( \frac{1+\lambda^2}{\lambda^2} - \frac{2}{\pi} \right)^2 \cdot \left( \frac{2\lambda}{\lambda^2} - \frac{2(1+\lambda^2)}{\lambda^3} \right) \\
  &= \left( \frac{\pi^3}{2(4-\pi)^2} \right) \cdot 3 \left( \frac{1+\lambda^2}{\lambda^2} - \frac{2}{\pi} \right)^2 \cdot
    \left( \cancel{\frac{2}{\lambda}} - \frac{2}{\lambda^3} - \cancel{\frac{2}{\lambda}} \right) \\
  &= \underbrace{\left( \frac{\pi^3}{2(4-\pi)^2} \right) \cdot 3 \left( \frac{1+\lambda^2}{\lambda^2} - \frac{2}{\pi} \right)^2}_\textnormal{Always positive} \quad \cdot \quad
     \underbrace{\left( -\frac{2}{\lambda^3} \right)}_{\mathclap{\textnormal{Negative when $\lambda > 0$}}}.
\end{align*}

This convenient fact allows us to find lower and upper bounds for $\lambda$ and
repeatedly bisect our interval until we are within $e$ of $k_{n,p}$.

(For the following calculations, it is helpful to keep in mind that because $f$
is decreasing in $\lambda$, smaller values of $\lambda$ will produce larger
values of $f$, and vice versa.)
    
\begin{enumerate}
  \item \textit{Find $k_{n,p}$.}
  
    Our value: $\displaystyle k_{n,p} = \frac{25 \cdot 0.1 \cdot 0.9}{(1 - 2\cdot0.1)^2} = 3.5156$.

  \item \textit{Find $a$ and $b$ such that $f(a) > k_{n,p} > f(b)$.}
    
    Our values: $a=1$, $b=3$.
  
  \item \textit{Repeatedly bisect $(a,b)$ until $f(c)$ is within $e$ of $k_{n,p}$.}

    Calculate $c=\frac{a+b}{2}$.

    \begin{itemize}
      \item If $f(c) \leq k_{n,p} - 0.01$, we need a small value of $c$, so we
      take our new interval to be $(a, c)$.
    
      \item If $f(c) \geq k_{n,p} + 0.01$, we need a larger value of $c$, so we
      take our new interval to be $(c, b)$.
    \end{itemize}
    
    Repeat this step until $f(c)$ is within $e$ of $k_{n,p}$, or more precisely
    $k_{n,p} - 0.01 < f(c) < k_{n,p} + 0.01$.

    The following table shows our iterations:

    \renewcommand{\arraystretch}{1.3}
    \begin{center}
      \begin{tabular}{c c c c c c c}
        Iteration & $a$  & $b$   & $c$    & $f(c)$ & $f(c) \leq k_{n,p} - 0.01$ & $f(c) \geq k_{n,p} + 0.01$ \\
        \hline
        1         & 2.00 & 3.000 & 2.5000 & 3.0164 & True                       & False \\
        2         & 2.00 & 2.500 & 2.2500 & 3.7129 & False                      & True \\
        3         & 2.25 & 2.500 & 2.3750 & 3.3252 & True                       & False \\
        4         & 2.25 & 2.375 & 2.3125 & 3.5076 & False                      & False \\
      \end{tabular}
    \end{center}

    We take the last value of $c$: 2.3125.

  \item \textit{Find the sign of $(1-2p)$.}

    Our $p = 0.1 \Ra (1 - 2\cdot0.1) = 0.8 \Ra$ positive.

  \item \textit{Final answer: \{sign of $(1-2p)$\}$\lambda$}

    Our final answer: $\lambda = 2.3125$.
\end{enumerate}

Once we have $\lambda$, we can easily find $\sigma$:

\begin{equation*}
  \sigma = \sqrt{\frac{np(1-p)}{1 - \frac{2}{\pi} \cdot \frac{\lambda^2}{1 + \lambda^2}}} = \sqrt{\frac{25 \cdot 0.1 \cdot 0.9}{1 - \frac{2}{\pi} \cdot \frac{2.3125^2}{1 + 2.3125^2}}} = 2.2029. \\
\end{equation*}

And with $\lambda$ and $\sigma$, we can also find $\mu$:

\begin{equation*}
  \mu = np - \sigma \cdot \sqrt{\frac{2}{\pi}} \cdot \frac{\lambda}{\sqrt{1 + \lambda^2}} = 25 \cdot 0.1 - 2.2029 \cdot \sqrt{\frac{2}{\pi}} \cdot \frac{2.3125}{\sqrt{1 + 2.3125^2}} = 0.8867.
\end{equation*}

\clearpage
