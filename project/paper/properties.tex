\section{The Skew-Normal}
\label{sec:properties}

The skew-normal distribution is similar to the normal but with an added
parameter for skew that allows it to lean to the left or right. In this
section, we'll get to know some of its basic properties.

\subsection{Basics}

\begin{defn}
  Let $Y$ be a skew-normal distribution, with location parameter $\mu \in \R$,
  scale parameter $\sigma > 0$, and shape parameter $\lambda \in \R$; we will
  denote it $SN(\mu, \sigma, \lambda)$.\footnotemark Then $Y$ has pdf

  \begin{equation} \label{eq:sn-pdf}
    f(x|\mu, \sigma, \lambda) = \frac2\sigma \cdot \phi \left( \frac{x-\mu}{\sigma} \right) \cdot \Phi \left( \frac{\lambda(x-\mu)}{\sigma} \right), \quad x \in \R,
  \end{equation}

  where $\phi$ is the standard normal pdf and $\Phi$ is the standard normal
  cdf. \citet{main}
  
  \footnotetext{In this paper, we have followed the notation established by
  \citet{main} in naming our parameters $\mu$, $\sigma$, and $\lambda$. It does
  seem intuitive, after all, for the skew-normal to simply "extend" the normal
  with an extra parameter.
  
  An unfortunate consequence of this choice, however, is the misconception that
  $\mu$ and $\sigma$ are related to the mean and variance the way they are in
  the normal distribution. In the case of the skew-normal, it is more
  productive to think of them as abstract location and scale parameters. In
  fact, a quick comparison of figures \ref{fig:comparison-n25},
  \ref{fig:comparison-n50}, and \ref{fig:comparison-n100} and the corresponding
  parameters in Table \ref{tab:sn-approx-table} shows that $\mu$ and $\sigma$
  are not visually related to the curve in any intuitive way.}

\end{defn}

The skew-normal was first introduced by \citet{o'hagan} and was most notably
developed by \citet{azzalini-1985}. Some of its other basic properties, given
by \citet{pewsey}, are

\begin{align}
  E(Y) &= \mu + b \delta \sigma, \nonumber \\
  E(Y^2) &= \mu^2 + 2b \delta \mu \sigma + \sigma^2, \label{eq:sn-basic-properties} \\
  E(Y^3) &= \mu^3 + 3 b \delta \mu^2 \sigma + 3 \mu \sigma^2 + 3 b \delta \sigma^3 - b \delta^3 \sigma^3, \nonumber \\
  Var(Y) &= \sigma^2 (1 - b^2 \delta^2), \nonumber
\end{align}

where $b = \sqrt{\frac{2}{\pi}}$ and $\delta = \frac{\lambda}{\sqrt{1 +
\lambda^2}}$.

The $SN(0,1,\lambda)$ distribution is called the standard skew-normal; its pdf
is

\begin{equation} \label{eq:standard-sn-pdf}
  f(x|\lambda) = 2 \cdot \phi(x) \cdot \Phi (\lambda x), \quad x \in \R.
\end{equation}

Similar to the normal and standard normal, $Z = \frac{Y - \mu}{\sigma}$ and $Y
= \sigma Z + \mu$.

A natural question to ask is how the skew-normal relates to the normal.
Fortunately, the connection is very intuitive: When $\lambda = 0$, Equation
\eqref{eq:sn-pdf} becomes

\begin{align*}
  f(x|\lambda=0) &= \frac2\sigma \cdot \phi \left( \frac{x-\mu}{\sigma} \right) \cdot \Phi(0) \\
  &= \frac2\sigma \cdot \phi \left( \frac{x-\mu}{\sigma} \right) \cdot 0.5 \\
  &= \frac1\sigma \cdot \phi \left( \frac{x-\mu}{\sigma} \right) \\
  &= \frac1\sigma \cdot \frac{1}{\sqrt{2\pi}} \;\cdot\; \exp \left( -\frac{(x-\mu)^2}{2\sigma^2} \right) \\
  &= \frac{1}{\sqrt{2\pi}\sigma} \;\cdot\; \exp \left( -\frac{(x-\mu)^2}{2\sigma^2} \right),
\end{align*}

which is the pdf of the normal distribution. Furthermore, when $\lambda > 0$,
the curve skews to the left, and when $\lambda < 0$, it skews to the right (a
property we will prove in section \ref{subsec:four-properties}).

\subsection{Four Properties}
\label{subsec:four-properties}

The following four properties of the skew-normal, given by \citet{main}, help
shed light on our enigmatic new distribution:

\begin{property} \label{prop:1}
  If $Z \sim SN(0, 1, \lambda)$, then $(-Z) \sim SN(0, 1, -\lambda)$.
\end{property}

\begin{proof}
  The standard normal pdf is an even function: $\phi(-x) =
  \frac{1}{\sqrt{2\pi}}\;e^{-(-x)^2/2} = \frac{1}{\sqrt{2\pi}}\;e^{-x^2/2} =
  \phi(x)$. But the standard normal cdf, \thinspace $\Phi(x) = \int_{-\infty}^x
  \phi(t)\;dt$, \thinspace is not even, being 0 near $-\infty$ and 1 near
  $\infty$. Thus,
  
  \begin{align*}
    f_{(-Z)}(x) &= f_Z(-x) \\
    & = 2 \cdot \phi(-x) \cdot \Phi (-\lambda x) \\
    & = 2 \cdot \phi(x) \cdot \Phi (-\lambda x),
  \end{align*}

  which is the pdf of $SN(0, 1, -\lambda)$.
\end{proof}

\begin{property} \label{prop:2}
  If $Z \sim SN(0, 1, \lambda)$, then $Z^2 \sim \chi^2_1$ (chi-square with 1 degree of freedom).
\end{property}

\begin{proof}
  To prove our result, we make use of Lemma 1 in \citet{azzalini}, which we
  restate here:

  \begin{helper-lem}
    If $f_0$ is a one-dimensional probability density function symmetric about
    0, and $G$ is a one-dimensional distribution function such that $G'$ exists
    and is a density symmetric about 0, then

    \begin{equation}
      \label{eq:azzalini-perturbation-invariance}
      f(z) = 2 \cdot f_0(z) \cdot G\{w(z)\} \quad (-\infty < z < \infty)
    \end{equation}

    is a density function for any odd function $w(\cdot)$.
  \end{helper-lem}

  This lemma provides a very useful corollary:

  \begin{helper-cor}[Perturbation Invariance]
    If $Y \sim f_0$ and $Z \sim f$, then $|Y| \overset{d}{=} |Z|$, where the
    notation $\overset{d}{=}$ denotes equality in distribution.    
  \end{helper-cor}

  Let $f_0 = \phi$ and $G = \Phi$. Then, $f_Z(z) = 2 \cdot \phi(z) \cdot
  \Phi(\lambda z)$ conforms to Equation
  \eqref{eq:azzalini-perturbation-invariance}, and we can conclude that $\phi$
  and $Z$ are equal in distribution.

  We will now show that $\phi^2 \sim \chi^2_1$ by deriving its moment
  generating function (mgf):\footnote{If $X$ is a random variable, then the
  expected value $M_X(t) = E(e^{tX})$ is called the moment generating function
  (mgf) of X if this expected value exists for all values of $t$ in some
  interval of the form $-h < t < h$ for some $h > 0$. Definition 2.5.1,
  \citet{textbook}.}

  \begin{align*}
    M_{\phi^2}(t) &= E[e^{tx^2}] \\
    &= \int_{-\infty}^\infty \; e^{tx^2} \left[ \frac{1}{\sqrt{2\pi}} e^{-x^2/2} \right] dx \\
    &= \int_{-\infty}^\infty \; \frac{1}{\sqrt{2\pi}} \; e^{tx^2 - x^2/2} \; dx \\
    &= \int_{-\infty}^\infty \; \frac{1}{\sqrt{2\pi}} \; e^{-\frac{x^2}{2}(1 - 2t)} \; dx \\
    &= \int_{-\infty}^\infty \; \frac{1}{\sqrt{2\pi}} \; e^{-\frac{1}{2}(\sqrt{1-2t} \; x)^2} \; dx \;; \\
    \intertext{let $u = (\sqrt{1-2t}) \, x$; then $du = (\sqrt{1-2t}) \, dx$, \enspace $dx = \frac{du}{\sqrt{1-2t}}$, \enspace and our limits become $x \to -\infty, x \to \infty \Ra
      u \to -\infty, u \to \infty$:}
    &= \int_{-\infty}^\infty \; \frac{1}{\sqrt{2\pi}} \; e^{-u^2/2} \; \left( \frac{1}{\sqrt{1-2t}} du \right) \\
    &= \frac{1}{\sqrt{1-2t}} \; \underbrace{\left( \int_{-\infty}^\infty \; \frac{1}{\sqrt{2\pi}} \; e^{-u^2/2} \; du \right)}_{\mathclap{\textnormal{$\phi(u)$ integrated over
      $(-\infty,\infty)$ = 1}}} \label{phi-pdf} \\
    &= \frac{1}{\sqrt{1-2t}} \;,
  \end{align*}

  which is the MGF of the $\chi^2_1$. Since $Z$ is equal in distribution to
  $\phi$, we can also conclude that $Z^2 \sim \chi^2_1$. \end{proof}

\begin{property} \label{prop:3}
  As $\lambda \to \pm \infty$, \thinspace $SN(0,1,\lambda)$ tends to the half normal distribution, $\pm |N(0,1)|$.
\end{property}

To prove our theorem, it is helpful to formally define the half normal distribution:

\begin{helper-lem} \label{lem:p2-half-normal}
  Let $X \sim N(0, \sigma^2)$. Then the distribution of $|X|$ is a half-normal
  random variable with parameter $\sigma$ and

  \begin{equation*}
    f_{|X|}(x) =
    \begin{dcases*}
      0              & when $-\infty < x \leq 0$ \\
      2 \cdot f_X(x) & when $0 < x < \infty$ 
    \end{dcases*}
    .
  \end{equation*}
\end{helper-lem}

\begin{proof}
  Let $X \sim N(0, \sigma^2)$, defined over $A = (-\infty, \infty)$. Define 
  
  \begin{equation*}
    Y = |X| =
    \begin{dcases*}
      -x & if $x < 0$ \\
      0 & if $x = 0$ \\
      x & if $x > 0$ \\
    \end{dcases*}
    .
  \end{equation*}

  $Y$ is not one-to-one over $A$. However, we can partition $A$ into disjoint
  subsets $A_1 = (-\infty, 0)$, $A_2 = (0, \infty)$, and $A_3 = \{0\}$ such
  that $A = A_1 \cup A_2 \cup A_3$ and $Y$ is one-to-one over each $A_i$. We
  can then transform each piece separately using Theorem 6.3.2 from
  \citet{textbook}:\footnote{\textbf{Continuous transformations that are
  one-to-one}: Suppose that $X$ is a continuous random variable with pdf
  $f_X(x)$, and assume that $Y = u(X)$ defines a one-to-one transformation from
  $A = \{x|f_X(x) > 0\}$ on to $B = \{y|f_Y(y) > 0\}$ with inverse
  transformation $x = w(y)$. If the derivative $(d/dy)w(y)$ is continuous and
  nonzero on $B$, then the pdf of $Y$ is

  \begin{equation}
    \label{eq:transformations-1-to-1}
    f_Y(y) = f_X(w(y)) \left| \frac{d}{dy} w(y) \right| \qquad y \in B.
  \end{equation}

  Theorem 6.3.2, \citet{textbook}.
  \label{fnote:transformations-1-to-1}
  }

  On $A_1$: $y = -x \Ra x = -y$ and $\J = \left| \frac{dx}{dy} \right| = |-1|
  = 1$, yielding

  \begin{align*}
    f_Y(y) &= f_X(x) \cdot \J \\
    &= f_X(-y) \cdot 1 \\
    &= \frac{1}{\sqrt{2\pi}\sigma} \; e^{-\frac{(-y)^2}{2\sigma^2}} \\
    &= \frac{1}{\sqrt{2\pi}\sigma} \; e^{-\frac{y^2}{2\sigma^2}} \\
    &= f_X(y)
  \end{align*}

  over the domain $A_1: -\infty < x < 0 \Ra -\infty < -y < 0 \Ra 0 < y < \infty :B_1$.

  Similarly, on $A_2$: $y = x \Ra x = y$ and $\J = \left| \frac{dx}{dy}
  \right| = |1| = 1$, yielding

  \begin{align*}
    f_Y(y) &= f_X(x) \cdot \J \\
    &= f_X(y) \cdot 1 \\
    &= f_X(y)
  \end{align*}

  over the domain $A_2: 0 < x < \infty \Ra 0 < y < \infty :B_2$.

  On $A_3$, we have $x = 0 \Ra y = 0$ and $\J = \left| \frac{dx}{dy} \right| =
  |0| = 0$, yielding $f_Y(y) = f_X(x) \cdot \J = f_X(x) \cdot 0 = 0$.

  Then, by Equation 6.3.10 from \citet{textbook},\footnote{\textbf{Continuous
  transformations that are not one-to-one}: When $u(x)$ is not one-to-one over
  $A$, we can replace equation \eqref{eq:transformations-1-to-1} in footnote
  \ref{fnote:transformations-1-to-1} with 

  \begin{equation*}
    f_Y(y) = \sum_j f_X(w_j(y)) \left| \frac{d}{dy} w_j(y) \right|.
  \end{equation*}

  Equation 6.3.10, \citet{textbook}.
  }

  \begin{align*}
    f_Y(y) &= \{ f_Y(y) \textrm{ over } A_1 \} + \{ f_Y(y) \textrm{ over } A_2 \} \\
    &= f_X(y) + f_X(y) \\
    &= 2 \cdot f_X(y)
  \end{align*}

  over $B = B_1 \cup B_2 = (0, \infty)$, and 0 otherwise.
\end{proof}

With Lemma \ref{lem:p2-half-normal}, we can easily show our property:

\begin{proof}[Proof of Property 3]
  Let $Z \sim SN(0,1,\lambda)$. Recall that $f_Z(x) = 2 \cdot \phi(x) \cdot
  \Phi(\lambda x)$.

  Consider $\lim_{\lambda \to \infty} f_X(x)$. When $x$ is negative, $\lambda x
  \to -\infty$ and thus $\Phi(\lambda x) \to 0$. When $x$ is positive, however,
  $\lambda x \to \infty$ and $\Phi(\lambda x) \to 1$. Thus,

  \begin{equation}
    \label{eq:p2-positive-half-normal}
    \lim_{\lambda \to \infty} 2 \cdot \phi(x) \cdot \Phi(\lambda x) \eq
    \begin{dcases*}
      0 & when $x \leq 0$ \\
      2 \cdot \phi(x) & when $x > 0$
    \end{dcases*}
    \eq |N(0,1)|.
  \end{equation}

  In $\lim_{\lambda \to -\infty} f_X(x)$, the signs are reversed. When $x$ is
  negative, $\lambda x \to \infty$ and $\Phi(\lambda x) \to 1$. When $x$ is
  positive, $\lambda x \to -\infty$ and $\Phi(\lambda x) \to 0$. Thus,

  \begin{equation}
    \label{eq:p2-negative-half-normal}
    \lim_{\lambda \to -\infty} 2 \cdot \phi(x) \cdot \Phi(\lambda x) \eq
    \begin{dcases*}
      2 \cdot \phi(x) & when $x < 0$ \\
      0 & when $x \geq 0$
    \end{dcases*}
    \eq -|N(0,1)|.
  \end{equation}
\end{proof}


\begin{property} \label{prop:4}
  The MGF of $SN(0,1,\lambda)$ is

  \begin{equation} \label{eq:p4-sn-mgf}
    M(t|\lambda) = 2 \cdot \Phi (\delta t) \cdot e^{t^2/2},
  \end{equation}
    
  where $\delta = \frac{\lambda}{\sqrt{1 + \lambda^2}}$ and $t \in (-\infty, \infty)$.
\end{property}

\begin{proof}
  According to Equation 5 in \citet{azzalini}, the MGF of $SN(\mu, \sigma,
  \lambda)$ is

  \begin{equation*}
    M(t) \eq E\{e^{tY}\} \eq 2 \cdot \exp \left( \mu t + \frac{\sigma^2 t^2}{2} \right) \cdot \Phi(\delta \sigma t),
  \end{equation*}

  where $\delta = \frac{\lambda}{\sqrt{1 + \lambda^2}} \in (-1, 1)$. It follows
  that the MGF of the $SN(0, 1, \lambda)$ is

  \begin{equation*}
    2 \cdot \exp \left( 0 \cdot t + \frac{1 \cdot t^2}{2} \right) \cdot \Phi(\delta \cdot 1 \cdot t) \eq 2 \cdot e^{t^2/2} \cdot \Phi(\delta t).
  \end{equation*}
\end{proof}
