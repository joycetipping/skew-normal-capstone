% Calculating a Skew-Normal Approximation
\frame{
  \textit{Appendix}

  Calculating a skew-normal approximation
}
\frame{ \frametitle{Calculating a Skew-Normal Approximation}
  Although easier with a computer program, calculating estimates for $\mu$, $\sigma$, and $\lambda$ by hand is perfectly possible.
  \pause

  Here's an example!
}
\begin{frame}[t]{Calculating a Skew-Normal Approximation: Finding $\lambda$}
  % This frame needs to be top-aligned to prevent bumping when the braces appear.
  \vspace{1cm}
  By far the biggest battle is $\lambda$.
  \pause

  We'll use a simplified version of equation \eqref{eq:solving-for-lambda}:
  \pause
  \begin{equation} \label{eq:calculating-lambda}
    \only<+>{\left( \frac{1+\lambda^2}{\lambda^2} - \frac{2}{\pi} \right)^3 \left( \frac{\pi^3}{2(4-\pi)^2} \right) = \frac{np(1-p)}{(1-2p)^2}}
    \only<+>{\underbrace{ \left( \frac{1+\lambda^2}{\lambda^2} - \frac{2}{\pi} \right)^3 \left( \frac{\pi^3}{2(4-\pi)^2} \right) }_\text{$f(\lambda)$}
              = \frac{np(1-p)}{(1-2p)^2}}
    \only<+->{\underbrace{ \left( \frac{1+\lambda^2}{\lambda^2} - \frac{2}{\pi} \right)^3 \left( \frac{\pi^3}{2(4-\pi)^2} \right) }_\text{$f(\lambda)$}
              =
              \underbrace{ \frac{np(1-p)}{(1-2p)^2} }_\text{$k_{n,p}$}}
  \end{equation}

  \uncover<+->{The closed form solution to \eqref{eq:calculating-lambda} is pretty hideous, so we'll take a numerical approach.}

  \uncover<+->{Our goal is to find $\lambda$ such that $f(\lambda)$ is within a certain margin of error ($e$) of $k_{n,p}$.}
\end{frame}
\frame{ \frametitle{Calculating a Skew-Normal Approximation: Finding $\lambda$}
  Recall that the sign of $\lambda$ is determined independently of the value.
  \pause

  It is possible to show, by taking its derivative, that $f$ is monotonically decreasing for positive $\lambda$.
  \pause

  This convenient fact allows us to find lower and upper bounds for $\lambda$ and repeatedly bisect our interval until we are within $e$ of $k_{n,p}$.
}
\frame{ \frametitle{Calculating a Skew-Normal Approximation: Finding $\lambda$}
  For this demonstration, we will take $n=25$ and $p=0.1$.
  \pause

  Since we're doing this by hand, we'll take our error margin $e$ to be a modest 0.1.
}
\frame{ \frametitle{Calculating a Skew-Normal Approximation: Finding $\lambda$}
  \textit{Step 1}: Find $k_{n,p}$.
  \pause

  Our value: $\displaystyle k_{n,p} = \frac{25 \cdot 0.1 \cdot 0.9}{(1 - 2\cdot0.1)^2} = 3.5156$.
  \pause

  \vspace{1cm}
  \textit{Step 2}: Find $a$ and $b$ such that $f(a) > k_{n,p} > f(b)$.
  \pause

  Our values: $a=1$, $b=3$.
}
\frame{ \frametitle{Calculating a Skew-Normal Approximation: Finding $\lambda$}
  \textit{Step 3}: Repeatedly bisect $(a,b)$ until $f(c)$ is within $e$ of $k_{n,p}$.
  \pause

  Calculate $c=\frac{a+b}{2}$.
  \pause
  \begin{itemize}[<+->]
    \item If $f(c) \leq k_{n,p} - 0.01$, we need a small value of $c$, so we take our new interval to be $(a, c)$.
    \item If $f(c) \geq k_{n,p} + 0.01$, we need a larger value of $c$, so we take our new interval to be $(c, b)$.
  \end{itemize}

  \uncover<+->{Repeat this step until $f(c)$ is within $e$ of $k_{n,p}$, or more precisely $k_{n,p} - 0.01 < f(c) < k_{n,p} + 0.01$.}
}
\frame{ \frametitle{Calculating a Skew-Normal Approximation: Finding $\lambda$}
  \textit{(Step 3)}

  The following table shows our iterations:
  \renewcommand{\arraystretch}{1.3}
  \begin{center}
    \scriptsize
    \begin{tabular}{c c c c c c c}
      \uncover<+->{Iteration & $a$  & $b$   & $c$    & $f(c)$ & $f(c) \leq k_{n,p} - 0.01$ & $f(c) \geq k_{n,p} + 0.01$ \\
                   \hline}
      \uncover<+->{1         & 2.00 & 3.000 & 2.5000 & 3.0164 & True                       & False \\}
      \uncover<+->{2         & 2.00 & 2.500 & 2.2500 & 3.7129 & False                      & True \\}
      \uncover<+->{3         & 2.25 & 2.500 & 2.3750 & 3.3252 & True                       & False \\}
      \uncover<+->{4         & 2.25 & 2.375 & 2.3125 & 3.5076 & False                      & False \\}
    \end{tabular}
  \end{center}

  \uncover<+->{We take the last value of $c$: 2.3125.}
}
\frame{ \frametitle{Calculating a Skew-Normal Approximation: Finding $\lambda$}
  \textit{Step 5}: Find the sign of $(1-2p)$.
  \pause

  Our $p = 0.1 \Ra (1 - 2\cdot0.1) = 0.8 \Ra$ positive.
  \pause

  \vspace{1cm}
  \textit{Step 6}: Final answer: \{sign of $(1-2p)$\}$\lambda$
  \pause

  Our final answer: $\lambda = 2.3125$.
}
\frame{ \frametitle{Calculating a Skew-Normal Approximation: Finding $\sigma$}
  Once we have $\lambda$, we can easily find $\sigma$:
  \pause
  \begin{equation*}
    \sigma = \sqrt{\frac{np(1-p)}{1 - \frac{2}{\pi} \cdot \frac{\lambda^2}{1 + \lambda^2}}} = \sqrt{\frac{25 \cdot 0.1 \cdot 0.9}{1 - \frac{2}{\pi} \cdot \frac{2.3125^2}{1 + 2.3125^2}}} = 2.2029.
  \end{equation*}
}
\frame{ \frametitle{Calculating a Skew-Normal Approximation: Finding $\mu$}
  And with $\lambda$ and $\sigma$, we can also find $\mu$:
  \pause
  \begin{align*}
    \mu &= np - \sigma \cdot \sqrt{\frac{2}{\pi}} \cdot \frac{\lambda}{\sqrt{1 + \lambda^2}} \\
    &= 25 \cdot 0.1 - 2.2029 \cdot \sqrt{\frac{2}{\pi}} \cdot \frac{2.3125}{\sqrt{1 + 2.3125^2}} \\
    &= 0.8867.
  \end{align*}
}
