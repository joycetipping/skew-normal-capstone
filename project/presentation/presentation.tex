\documentclass{beamer}
\usepackage{jmath}

\usetheme{default}
\usefonttheme{structuresmallcapsserif}
\setbeamertemplate{blocks}[rounded][shadow=true]

\newcommand{\vsep}{\vspace{0.2cm}}

\begin{document}
\title[The Skew-Normal Approx of the Binomial]{The Skew-Normal Approximation\\of the Binomial Distribution}
\author{Joyce Tipping}
\institute{Truman State University}
\date{Spring 2011}


\frame{\titlepage}

%% Introduction
\frame{
  \frametitle{Introduction}
  $X \sim Bin(n,p)$ where $0<p<1$ and $n = 1, 2, 3, \ldots$

  \pause
  \begin{align*}
    \uncover<2->{f_X(x) &= \binom{n}{x} \; p^x q^{n-x} \\}
    \uncover<3->{F_X(x) &= P(X \leq x) = \sum_{k=0}^x f_X(k)}
  \end{align*}
}
\frame{
  \frametitle{Introduction}
  The binomial cdf is easy to calculate for small $n$ ...

  \pause
  But as $n$ gets larger, it becomes increasingly difficult.

  \pause
  For example ...
}
\frame{
  \frametitle{Introduction}
  When $n=3$,
  \begin{equation*}
    F(1) = \binom{3}{1} \; p^1 q^2 + \binom{3}{0} \; p^0 q^3
  \end{equation*}

  \pause
  When $n=25$,
  \begin{align*}
    F(12) =& \binom{25}{12} \; p^{12} q^{13} + \binom{25}{11} \; p^{11} q^{14} + \binom{25}{10} \; p^{10} q^{15} + \binom{25}{9} \; p^9 q^{16} \\
          +& \binom{25}{8} \; p^8 q^{17} + \binom{25}{7} \; p^7 q^{18} + \binom{25}{6} \; p^6 q^{19} + \binom{25}{5} \; p^5 q^{20} \\
          +& \binom{25}{4} \; p^4 q^{21} + \binom{25}{3} \; p^3 q^{22} + \binom{25}{2} \; p^2 q^{23} + \binom{25}{1} \; p^1 q^{24} \\
          +& \binom{25}{0} \; p^0 q^{25}
  \end{align*}
}
\frame{
  \frametitle{Introduction}
  Normal Approximation of the Binomial:

  \begin{equation*}
    F_X(x) \approx \Phi \left( \frac{x + 0.5 - \mu}{\sigma} \right),
  \end{equation*}

  where $\mu = np$, $\sigma = \sqrt{np(1-p)}$, and $\Phi$ is the standard
  normal cdf.

  \vsep
  When does this work well? ... \pause In a nutshell, when the binomial is symmetric.
}
\frame{
  \frametitle{Introduction}

  The binomial is symmetric when
  \only<1->{$p=0.5$}
  \only<2>{or $n$ is very large.}

  \begin{center}
    \only<1>{\includegraphics[width=\textwidth]{../images/binomial-normal-1.png}}
    \only<2>{\includegraphics[width=\textwidth]{../images/binomial-normal-2.png}}
  \end{center}
}
\frame{
  \frametitle{Introduction}

  However, when $n$ is medium and $p$ is extreme ...

  \begin{center}
    \only<1>{\includegraphics[width=\textwidth]{../images/binomial-normal-sn-1.png}}
    \only<2-3>{\includegraphics[width=\textwidth]{../images/binomial-normal-sn-2.png}}
    \only<4>{\includegraphics[width=\textwidth]{../images/binomial-normal-sn-3.png}}
  \end{center}

  \only<1>{the binomial is very skewed.}
  \only<2>{the normal approximation doesn't work very well.}
  \only<3->{\textbf{Can we do better?}}
  \only<4>{Introducing ... the skew-normal distribution.}
}

%% Outline
\frame{
  \frametitle{Outline}
  What we're going to cover:
  \pause
  \begin{enumerate}[<+->]
    \item Skew-Normal distribution -- basic properties
    \item Method of Moments -- derive an approximation
    \item Accuracy -- compare to the normal approximation
  \end{enumerate}
}

%% The Skew-Normal Distribution
\frame{
  \frametitle{The Skew-Normal Distribution}
  \begin{definition}[Skew-normal]
    Let $Y$ be a skew-normal distribution, with location parameter $\mu \in
    \R$, scale parameter $\sigma > 0$, and shape parameter $\lambda \in \R$.
    Then $Y$ has pdf

    \begin{equation*}
      f(x|\mu, \sigma, \lambda) = \frac2\sigma \cdot \phi \left( \frac{x-\mu}{\sigma} \right) \cdot \Phi \left( \frac{\lambda(x-\mu)}{\sigma} \right), \quad x \in \R,
    \end{equation*}

    where $\phi$ is the standard normal pdf and $\Phi$ is the standard normal
    cdf.

    \vsep
    We write $Y \sim SN(\mu, \sigma, \lambda)$.
  \end{definition}
}
\frame{
  \frametitle{The Skew-Normal Distribution}
  A few notes:
}
\frame{
  \frametitle{The Skew-Normal Distribution}
}
\frame{
  \frametitle{The Skew-Normal Distribution}
}
\frame{
  \frametitle{The Skew-Normal Distribution}
}
\frame{
  \frametitle{The Skew-Normal Distribution}
}
\frame{
  \frametitle{The Skew-Normal Distribution}
}
\end{document}
